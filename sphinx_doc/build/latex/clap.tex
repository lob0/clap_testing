% Generated by Sphinx.
\def\sphinxdocclass{report}
\documentclass[letterpaper,10pt,english]{sphinxmanual}
\usepackage[utf8]{inputenc}
\DeclareUnicodeCharacter{00A0}{\nobreakspace}
\usepackage{cmap}
\usepackage[T1]{fontenc}
\usepackage{babel}
\usepackage{times}
\usepackage[Bjarne]{fncychap}
\usepackage{longtable}
\usepackage{sphinx}
\usepackage{multirow}


\title{clap Documentation}
\date{January 14, 2015}
\release{0.1}
\author{Clap Development Team}
\newcommand{\sphinxlogo}{}
\renewcommand{\releasename}{Release}
\makeindex

\makeatletter
\def\PYG@reset{\let\PYG@it=\relax \let\PYG@bf=\relax%
    \let\PYG@ul=\relax \let\PYG@tc=\relax%
    \let\PYG@bc=\relax \let\PYG@ff=\relax}
\def\PYG@tok#1{\csname PYG@tok@#1\endcsname}
\def\PYG@toks#1+{\ifx\relax#1\empty\else%
    \PYG@tok{#1}\expandafter\PYG@toks\fi}
\def\PYG@do#1{\PYG@bc{\PYG@tc{\PYG@ul{%
    \PYG@it{\PYG@bf{\PYG@ff{#1}}}}}}}
\def\PYG#1#2{\PYG@reset\PYG@toks#1+\relax+\PYG@do{#2}}

\expandafter\def\csname PYG@tok@gd\endcsname{\def\PYG@tc##1{\textcolor[rgb]{0.63,0.00,0.00}{##1}}}
\expandafter\def\csname PYG@tok@gu\endcsname{\let\PYG@bf=\textbf\def\PYG@tc##1{\textcolor[rgb]{0.50,0.00,0.50}{##1}}}
\expandafter\def\csname PYG@tok@gt\endcsname{\def\PYG@tc##1{\textcolor[rgb]{0.00,0.27,0.87}{##1}}}
\expandafter\def\csname PYG@tok@gs\endcsname{\let\PYG@bf=\textbf}
\expandafter\def\csname PYG@tok@gr\endcsname{\def\PYG@tc##1{\textcolor[rgb]{1.00,0.00,0.00}{##1}}}
\expandafter\def\csname PYG@tok@cm\endcsname{\let\PYG@it=\textit\def\PYG@tc##1{\textcolor[rgb]{0.25,0.50,0.56}{##1}}}
\expandafter\def\csname PYG@tok@vg\endcsname{\def\PYG@tc##1{\textcolor[rgb]{0.73,0.38,0.84}{##1}}}
\expandafter\def\csname PYG@tok@m\endcsname{\def\PYG@tc##1{\textcolor[rgb]{0.13,0.50,0.31}{##1}}}
\expandafter\def\csname PYG@tok@mh\endcsname{\def\PYG@tc##1{\textcolor[rgb]{0.13,0.50,0.31}{##1}}}
\expandafter\def\csname PYG@tok@cs\endcsname{\def\PYG@tc##1{\textcolor[rgb]{0.25,0.50,0.56}{##1}}\def\PYG@bc##1{\setlength{\fboxsep}{0pt}\colorbox[rgb]{1.00,0.94,0.94}{\strut ##1}}}
\expandafter\def\csname PYG@tok@ge\endcsname{\let\PYG@it=\textit}
\expandafter\def\csname PYG@tok@vc\endcsname{\def\PYG@tc##1{\textcolor[rgb]{0.73,0.38,0.84}{##1}}}
\expandafter\def\csname PYG@tok@il\endcsname{\def\PYG@tc##1{\textcolor[rgb]{0.13,0.50,0.31}{##1}}}
\expandafter\def\csname PYG@tok@go\endcsname{\def\PYG@tc##1{\textcolor[rgb]{0.20,0.20,0.20}{##1}}}
\expandafter\def\csname PYG@tok@cp\endcsname{\def\PYG@tc##1{\textcolor[rgb]{0.00,0.44,0.13}{##1}}}
\expandafter\def\csname PYG@tok@gi\endcsname{\def\PYG@tc##1{\textcolor[rgb]{0.00,0.63,0.00}{##1}}}
\expandafter\def\csname PYG@tok@gh\endcsname{\let\PYG@bf=\textbf\def\PYG@tc##1{\textcolor[rgb]{0.00,0.00,0.50}{##1}}}
\expandafter\def\csname PYG@tok@ni\endcsname{\let\PYG@bf=\textbf\def\PYG@tc##1{\textcolor[rgb]{0.84,0.33,0.22}{##1}}}
\expandafter\def\csname PYG@tok@nl\endcsname{\let\PYG@bf=\textbf\def\PYG@tc##1{\textcolor[rgb]{0.00,0.13,0.44}{##1}}}
\expandafter\def\csname PYG@tok@nn\endcsname{\let\PYG@bf=\textbf\def\PYG@tc##1{\textcolor[rgb]{0.05,0.52,0.71}{##1}}}
\expandafter\def\csname PYG@tok@no\endcsname{\def\PYG@tc##1{\textcolor[rgb]{0.38,0.68,0.84}{##1}}}
\expandafter\def\csname PYG@tok@na\endcsname{\def\PYG@tc##1{\textcolor[rgb]{0.25,0.44,0.63}{##1}}}
\expandafter\def\csname PYG@tok@nb\endcsname{\def\PYG@tc##1{\textcolor[rgb]{0.00,0.44,0.13}{##1}}}
\expandafter\def\csname PYG@tok@nc\endcsname{\let\PYG@bf=\textbf\def\PYG@tc##1{\textcolor[rgb]{0.05,0.52,0.71}{##1}}}
\expandafter\def\csname PYG@tok@nd\endcsname{\let\PYG@bf=\textbf\def\PYG@tc##1{\textcolor[rgb]{0.33,0.33,0.33}{##1}}}
\expandafter\def\csname PYG@tok@ne\endcsname{\def\PYG@tc##1{\textcolor[rgb]{0.00,0.44,0.13}{##1}}}
\expandafter\def\csname PYG@tok@nf\endcsname{\def\PYG@tc##1{\textcolor[rgb]{0.02,0.16,0.49}{##1}}}
\expandafter\def\csname PYG@tok@si\endcsname{\let\PYG@it=\textit\def\PYG@tc##1{\textcolor[rgb]{0.44,0.63,0.82}{##1}}}
\expandafter\def\csname PYG@tok@s2\endcsname{\def\PYG@tc##1{\textcolor[rgb]{0.25,0.44,0.63}{##1}}}
\expandafter\def\csname PYG@tok@vi\endcsname{\def\PYG@tc##1{\textcolor[rgb]{0.73,0.38,0.84}{##1}}}
\expandafter\def\csname PYG@tok@nt\endcsname{\let\PYG@bf=\textbf\def\PYG@tc##1{\textcolor[rgb]{0.02,0.16,0.45}{##1}}}
\expandafter\def\csname PYG@tok@nv\endcsname{\def\PYG@tc##1{\textcolor[rgb]{0.73,0.38,0.84}{##1}}}
\expandafter\def\csname PYG@tok@s1\endcsname{\def\PYG@tc##1{\textcolor[rgb]{0.25,0.44,0.63}{##1}}}
\expandafter\def\csname PYG@tok@gp\endcsname{\let\PYG@bf=\textbf\def\PYG@tc##1{\textcolor[rgb]{0.78,0.36,0.04}{##1}}}
\expandafter\def\csname PYG@tok@sh\endcsname{\def\PYG@tc##1{\textcolor[rgb]{0.25,0.44,0.63}{##1}}}
\expandafter\def\csname PYG@tok@ow\endcsname{\let\PYG@bf=\textbf\def\PYG@tc##1{\textcolor[rgb]{0.00,0.44,0.13}{##1}}}
\expandafter\def\csname PYG@tok@sx\endcsname{\def\PYG@tc##1{\textcolor[rgb]{0.78,0.36,0.04}{##1}}}
\expandafter\def\csname PYG@tok@bp\endcsname{\def\PYG@tc##1{\textcolor[rgb]{0.00,0.44,0.13}{##1}}}
\expandafter\def\csname PYG@tok@c1\endcsname{\let\PYG@it=\textit\def\PYG@tc##1{\textcolor[rgb]{0.25,0.50,0.56}{##1}}}
\expandafter\def\csname PYG@tok@kc\endcsname{\let\PYG@bf=\textbf\def\PYG@tc##1{\textcolor[rgb]{0.00,0.44,0.13}{##1}}}
\expandafter\def\csname PYG@tok@c\endcsname{\let\PYG@it=\textit\def\PYG@tc##1{\textcolor[rgb]{0.25,0.50,0.56}{##1}}}
\expandafter\def\csname PYG@tok@mf\endcsname{\def\PYG@tc##1{\textcolor[rgb]{0.13,0.50,0.31}{##1}}}
\expandafter\def\csname PYG@tok@err\endcsname{\def\PYG@bc##1{\setlength{\fboxsep}{0pt}\fcolorbox[rgb]{1.00,0.00,0.00}{1,1,1}{\strut ##1}}}
\expandafter\def\csname PYG@tok@kd\endcsname{\let\PYG@bf=\textbf\def\PYG@tc##1{\textcolor[rgb]{0.00,0.44,0.13}{##1}}}
\expandafter\def\csname PYG@tok@ss\endcsname{\def\PYG@tc##1{\textcolor[rgb]{0.32,0.47,0.09}{##1}}}
\expandafter\def\csname PYG@tok@sr\endcsname{\def\PYG@tc##1{\textcolor[rgb]{0.14,0.33,0.53}{##1}}}
\expandafter\def\csname PYG@tok@mo\endcsname{\def\PYG@tc##1{\textcolor[rgb]{0.13,0.50,0.31}{##1}}}
\expandafter\def\csname PYG@tok@mi\endcsname{\def\PYG@tc##1{\textcolor[rgb]{0.13,0.50,0.31}{##1}}}
\expandafter\def\csname PYG@tok@kn\endcsname{\let\PYG@bf=\textbf\def\PYG@tc##1{\textcolor[rgb]{0.00,0.44,0.13}{##1}}}
\expandafter\def\csname PYG@tok@o\endcsname{\def\PYG@tc##1{\textcolor[rgb]{0.40,0.40,0.40}{##1}}}
\expandafter\def\csname PYG@tok@kr\endcsname{\let\PYG@bf=\textbf\def\PYG@tc##1{\textcolor[rgb]{0.00,0.44,0.13}{##1}}}
\expandafter\def\csname PYG@tok@s\endcsname{\def\PYG@tc##1{\textcolor[rgb]{0.25,0.44,0.63}{##1}}}
\expandafter\def\csname PYG@tok@kp\endcsname{\def\PYG@tc##1{\textcolor[rgb]{0.00,0.44,0.13}{##1}}}
\expandafter\def\csname PYG@tok@w\endcsname{\def\PYG@tc##1{\textcolor[rgb]{0.73,0.73,0.73}{##1}}}
\expandafter\def\csname PYG@tok@kt\endcsname{\def\PYG@tc##1{\textcolor[rgb]{0.56,0.13,0.00}{##1}}}
\expandafter\def\csname PYG@tok@sc\endcsname{\def\PYG@tc##1{\textcolor[rgb]{0.25,0.44,0.63}{##1}}}
\expandafter\def\csname PYG@tok@sb\endcsname{\def\PYG@tc##1{\textcolor[rgb]{0.25,0.44,0.63}{##1}}}
\expandafter\def\csname PYG@tok@k\endcsname{\let\PYG@bf=\textbf\def\PYG@tc##1{\textcolor[rgb]{0.00,0.44,0.13}{##1}}}
\expandafter\def\csname PYG@tok@se\endcsname{\let\PYG@bf=\textbf\def\PYG@tc##1{\textcolor[rgb]{0.25,0.44,0.63}{##1}}}
\expandafter\def\csname PYG@tok@sd\endcsname{\let\PYG@it=\textit\def\PYG@tc##1{\textcolor[rgb]{0.25,0.44,0.63}{##1}}}

\def\PYGZbs{\char`\\}
\def\PYGZus{\char`\_}
\def\PYGZob{\char`\{}
\def\PYGZcb{\char`\}}
\def\PYGZca{\char`\^}
\def\PYGZam{\char`\&}
\def\PYGZlt{\char`\<}
\def\PYGZgt{\char`\>}
\def\PYGZsh{\char`\#}
\def\PYGZpc{\char`\%}
\def\PYGZdl{\char`\$}
\def\PYGZhy{\char`\-}
\def\PYGZsq{\char`\'}
\def\PYGZdq{\char`\"}
\def\PYGZti{\char`\~}
% for compatibility with earlier versions
\def\PYGZat{@}
\def\PYGZlb{[}
\def\PYGZrb{]}
\makeatother

\begin{document}

\maketitle
\tableofcontents
\phantomsection\label{index::doc}


Contents:


\chapter{CLAP's Tutorial}
\label{tutorial::doc}\label{tutorial:clap-s-tutorial}\label{tutorial:welcome-to-clap-s-documentation}

\section{Get yourself acquainted with CLAP project.}
\label{tutorial:get-yourself-acquainted-with-clap-project}
Take a brief tour


\chapter{Project Summary}
\label{project::doc}\label{project:project-summary}

\section{Goals}
\label{project:goals}
to be completed..


\section{Sponsors and Partnerships}
\label{project:sponsors-and-partnerships}
to be completed..


\chapter{restaurants package}
\label{restaurants:restaurants-package}\label{restaurants::doc}

\section{Subpackages}
\label{restaurants:subpackages}

\subsection{restaurants.migrations package}
\label{restaurants.migrations::doc}\label{restaurants.migrations:restaurants-migrations-package}

\subsubsection{Submodules}
\label{restaurants.migrations:submodules}

\subsubsection{restaurants.migrations.0001\_initial module}
\label{restaurants.migrations:restaurants-migrations-0001-initial-module}

\subsubsection{restaurants.migrations.0002\_auto\_\_del\_minor\_loca\_\_add\_loca\_main\_\_add\_field\_minor\_restaurant\_Rstr\_i module}
\label{restaurants.migrations:restaurants-migrations-0002-auto-del-minor-loca-add-loca-main-add-field-minor-restaurant-rstr-i-module}

\subsubsection{restaurants.migrations.0003\_auto\_\_chg\_field\_loca\_main\_Loca\_lat\_\_chg\_field\_loca\_main\_Loca\_long module}
\label{restaurants.migrations:restaurants-migrations-0003-auto-chg-field-loca-main-loca-lat-chg-field-loca-main-loca-long-module}

\subsubsection{restaurants.migrations.0004\_pass module}
\label{restaurants.migrations:restaurants-migrations-0004-pass-module}

\subsubsection{Module contents}
\label{restaurants.migrations:module-contents}\label{restaurants.migrations:module-restaurants.migrations}\index{restaurants.migrations (module)}

\section{Submodules}
\label{restaurants:submodules}

\section{restaurants.admin module}
\label{restaurants:restaurants-admin-module}

\section{restaurants.crypto module}
\label{restaurants:module-restaurants.crypto}\label{restaurants:restaurants-crypto-module}\index{restaurants.crypto (module)}
Cripto.py works as a library
\begin{description}
\item[{Classes:}] \leavevmode
PKCS7Encoder
InvalidBlockSizeError

\end{description}
\index{PKCS7Encoder (class in restaurants.crypto)}

\begin{fulllineitems}
\phantomsection\label{restaurants:restaurants.crypto.PKCS7Encoder}\pysiglinewithargsret{\strong{class }\code{restaurants.crypto.}\bfcode{PKCS7Encoder}}{\emph{block\_size=16}}{}
Technique for padding a string as defined in RFC 2315, section 10.3, note \#2
\index{PKCS7Encoder.InvalidBlockSizeError}

\begin{fulllineitems}
\phantomsection\label{restaurants:restaurants.crypto.PKCS7Encoder.InvalidBlockSizeError}\pysigline{\strong{exception }\bfcode{InvalidBlockSizeError}}
Bases: \href{http://docs.python.org/library/exceptions.html\#exceptions.Exception}{\code{exceptions.Exception}}

Raised for invalid block sizes

\end{fulllineitems}

\index{encode() (restaurants.crypto.PKCS7Encoder method)}

\begin{fulllineitems}
\phantomsection\label{restaurants:restaurants.crypto.PKCS7Encoder.encode}\pysiglinewithargsret{\code{PKCS7Encoder.}\bfcode{encode}}{\emph{text}}{}
\end{fulllineitems}

\index{decode() (restaurants.crypto.PKCS7Encoder method)}

\begin{fulllineitems}
\phantomsection\label{restaurants:restaurants.crypto.PKCS7Encoder.decode}\pysiglinewithargsret{\code{PKCS7Encoder.}\bfcode{decode}}{\emph{text}}{}
\end{fulllineitems}


\end{fulllineitems}

\index{sha256() (in module restaurants.crypto)}

\begin{fulllineitems}
\phantomsection\label{restaurants:restaurants.crypto.sha256}\pysiglinewithargsret{\code{restaurants.crypto.}\bfcode{sha256}}{\emph{x}}{}
Computes the SHA256 checksum of the given data

\end{fulllineitems}

\index{sha512() (in module restaurants.crypto)}

\begin{fulllineitems}
\phantomsection\label{restaurants:restaurants.crypto.sha512}\pysiglinewithargsret{\code{restaurants.crypto.}\bfcode{sha512}}{\emph{x}}{}
Computes the SHA512 checksum of the given data

\end{fulllineitems}

\index{cipher() (in module restaurants.crypto)}

\begin{fulllineitems}
\phantomsection\label{restaurants:restaurants.crypto.cipher}\pysiglinewithargsret{\code{restaurants.crypto.}\bfcode{cipher}}{\emph{k}, \emph{m}, \emph{iv}}{}
Encrypts a message

\end{fulllineitems}

\index{decipher() (in module restaurants.crypto)}

\begin{fulllineitems}
\phantomsection\label{restaurants:restaurants.crypto.decipher}\pysiglinewithargsret{\code{restaurants.crypto.}\bfcode{decipher}}{\emph{k}, \emph{ct}, \emph{iv}}{}
Decrypts a ciphertext

\end{fulllineitems}

\index{hmac() (in module restaurants.crypto)}

\begin{fulllineitems}
\phantomsection\label{restaurants:restaurants.crypto.hmac}\pysiglinewithargsret{\code{restaurants.crypto.}\bfcode{hmac}}{\emph{k}, \emph{x}}{}
Authenticates some value

\end{fulllineitems}

\index{verifies() (in module restaurants.crypto)}

\begin{fulllineitems}
\phantomsection\label{restaurants:restaurants.crypto.verifies}\pysiglinewithargsret{\code{restaurants.crypto.}\bfcode{verifies}}{\emph{k}, \emph{x}, \emph{tag}}{}
Verifies the MAC

\end{fulllineitems}

\index{encrypt() (in module restaurants.crypto)}

\begin{fulllineitems}
\phantomsection\label{restaurants:restaurants.crypto.encrypt}\pysiglinewithargsret{\code{restaurants.crypto.}\bfcode{encrypt}}{\emph{k1}, \emph{k2}, \emph{m}, \emph{iv}}{}
Encrypts under the encrypt-then-mac assumption

\end{fulllineitems}

\index{decrypt() (in module restaurants.crypto)}

\begin{fulllineitems}
\phantomsection\label{restaurants:restaurants.crypto.decrypt}\pysiglinewithargsret{\code{restaurants.crypto.}\bfcode{decrypt}}{\emph{k1}, \emph{k2}, \emph{iv}, \emph{ct}, \emph{tag}}{}
Decrypts under the encrypt-then-mac assumption

\end{fulllineitems}



\section{restaurants.key\_exchange module}
\label{restaurants:module-restaurants.key_exchange}\label{restaurants:restaurants-key-exchange-module}\index{restaurants.key\_exchange (module)}\index{PKCS7Encoder (class in restaurants.key\_exchange)}

\begin{fulllineitems}
\phantomsection\label{restaurants:restaurants.key_exchange.PKCS7Encoder}\pysiglinewithargsret{\strong{class }\code{restaurants.key\_exchange.}\bfcode{PKCS7Encoder}}{\emph{block\_size=16}}{}
Technique for padding a string as defined in RFC 2315, section 10.3, note \#2

The argument is the block size used by the encoder (in bytes)
\index{PKCS7Encoder.InvalidBlockSizeError}

\begin{fulllineitems}
\phantomsection\label{restaurants:restaurants.key_exchange.PKCS7Encoder.InvalidBlockSizeError}\pysigline{\strong{exception }\bfcode{InvalidBlockSizeError}}
Bases: \href{http://docs.python.org/library/exceptions.html\#exceptions.Exception}{\code{exceptions.Exception}}

Raised for invalid block sizes

\end{fulllineitems}

\index{encode() (restaurants.key\_exchange.PKCS7Encoder method)}

\begin{fulllineitems}
\phantomsection\label{restaurants:restaurants.key_exchange.PKCS7Encoder.encode}\pysiglinewithargsret{\code{PKCS7Encoder.}\bfcode{encode}}{\emph{text}}{}
The text is the STRING to be encoded

\end{fulllineitems}

\index{decode() (restaurants.key\_exchange.PKCS7Encoder method)}

\begin{fulllineitems}
\phantomsection\label{restaurants:restaurants.key_exchange.PKCS7Encoder.decode}\pysiglinewithargsret{\code{PKCS7Encoder.}\bfcode{decode}}{\emph{text}}{}
\end{fulllineitems}


\end{fulllineitems}

\index{sha256() (in module restaurants.key\_exchange)}

\begin{fulllineitems}
\phantomsection\label{restaurants:restaurants.key_exchange.sha256}\pysiglinewithargsret{\code{restaurants.key\_exchange.}\bfcode{sha256}}{\emph{x}}{}
Computes the SHA256 checksum of the given data, using the SHA256 library

\end{fulllineitems}

\index{sha512() (in module restaurants.key\_exchange)}

\begin{fulllineitems}
\phantomsection\label{restaurants:restaurants.key_exchange.sha512}\pysiglinewithargsret{\code{restaurants.key\_exchange.}\bfcode{sha512}}{\emph{x}}{}
Computes the SHA512 checksum of the given data, using the SHA512 library

\end{fulllineitems}

\index{cipher() (in module restaurants.key\_exchange)}

\begin{fulllineitems}
\phantomsection\label{restaurants:restaurants.key_exchange.cipher}\pysiglinewithargsret{\code{restaurants.key\_exchange.}\bfcode{cipher}}{\emph{k}, \emph{m}, \emph{iv}}{}
Ciphers a message `m' , with the key `k' and using the initialization vector `iv', using the AES library

\end{fulllineitems}

\index{decipher() (in module restaurants.key\_exchange)}

\begin{fulllineitems}
\phantomsection\label{restaurants:restaurants.key_exchange.decipher}\pysiglinewithargsret{\code{restaurants.key\_exchange.}\bfcode{decipher}}{\emph{k}, \emph{ct}, \emph{iv}}{}
Deciphers a ciphertext `ct', with the key `k' and using the initialization vector `iv', using the AES library

\end{fulllineitems}

\index{hmac() (in module restaurants.key\_exchange)}

\begin{fulllineitems}
\phantomsection\label{restaurants:restaurants.key_exchange.hmac}\pysiglinewithargsret{\code{restaurants.key\_exchange.}\bfcode{hmac}}{\emph{k}, \emph{x}}{}
Authenticates some value `x', with the key `k', using the HMAC library. I returns a value of 32 bytes

\end{fulllineitems}

\index{verifies() (in module restaurants.key\_exchange)}

\begin{fulllineitems}
\phantomsection\label{restaurants:restaurants.key_exchange.verifies}\pysiglinewithargsret{\code{restaurants.key\_exchange.}\bfcode{verifies}}{\emph{k}, \emph{x}, \emph{tag}}{}
Verifies the MAC (if the tag `tag' is correct for the value `x' with the key `k')

\end{fulllineitems}

\index{encrypt() (in module restaurants.key\_exchange)}

\begin{fulllineitems}
\phantomsection\label{restaurants:restaurants.key_exchange.encrypt}\pysiglinewithargsret{\code{restaurants.key\_exchange.}\bfcode{encrypt}}{\emph{k1}, \emph{k2}, \emph{m}, \emph{iv}}{}
Encrypts under the encrypt-then-mac assumption

\end{fulllineitems}

\index{decrypt() (in module restaurants.key\_exchange)}

\begin{fulllineitems}
\phantomsection\label{restaurants:restaurants.key_exchange.decrypt}\pysiglinewithargsret{\code{restaurants.key\_exchange.}\bfcode{decrypt}}{\emph{k1}, \emph{k2}, \emph{iv}, \emph{ct}, \emph{tag}}{}
Decrypts under the encrypt-then-mac assumption

\end{fulllineitems}

\index{parse\_ct() (in module restaurants.key\_exchange)}

\begin{fulllineitems}
\phantomsection\label{restaurants:restaurants.key_exchange.parse_ct}\pysiglinewithargsret{\code{restaurants.key\_exchange.}\bfcode{parse\_ct}}{\emph{ct}}{}
Parses a ciphertext and returns the iv, ciphered message and tag, (WARNING:) when the string has been converted to hexadecimal

\end{fulllineitems}

\index{parse\_ct\_original() (in module restaurants.key\_exchange)}

\begin{fulllineitems}
\phantomsection\label{restaurants:restaurants.key_exchange.parse_ct_original}\pysiglinewithargsret{\code{restaurants.key\_exchange.}\bfcode{parse\_ct\_original}}{\emph{ct}}{}
Parses a ciphertext when it is on its original form

\end{fulllineitems}

\index{computeSecret() (in module restaurants.key\_exchange)}

\begin{fulllineitems}
\phantomsection\label{restaurants:restaurants.key_exchange.computeSecret}\pysiglinewithargsret{\code{restaurants.key\_exchange.}\bfcode{computeSecret}}{\emph{g}, \emph{q}}{}
Computes the challenge to be used in the key exchange protocol

\end{fulllineitems}

\index{createChallenge() (in module restaurants.key\_exchange)}

\begin{fulllineitems}
\phantomsection\label{restaurants:restaurants.key_exchange.createChallenge}\pysiglinewithargsret{\code{restaurants.key\_exchange.}\bfcode{createChallenge}}{}{}
Computes the challenge to be used in the key exchange protocol

\end{fulllineitems}



\section{restaurants.models module}
\label{restaurants:restaurants-models-module}

\section{restaurants.test module}
\label{restaurants:module-restaurants.test}\label{restaurants:restaurants-test-module}\index{restaurants.test (module)}\index{test\_sum() (in module restaurants.test)}

\begin{fulllineitems}
\phantomsection\label{restaurants:restaurants.test.test_sum}\pysiglinewithargsret{\code{restaurants.test.}\bfcode{test\_sum}}{\emph{x}, \emph{y}}{}
\end{fulllineitems}



\section{restaurants.tests module}
\label{restaurants:restaurants-tests-module}\label{restaurants:module-restaurants.tests}\index{restaurants.tests (module)}\index{random\_string() (in module restaurants.tests)}

\begin{fulllineitems}
\phantomsection\label{restaurants:restaurants.tests.random_string}\pysiglinewithargsret{\code{restaurants.tests.}\bfcode{random\_string}}{\emph{n}}{}
Generates a random string with n characters

\end{fulllineitems}

\index{random\_text() (in module restaurants.tests)}

\begin{fulllineitems}
\phantomsection\label{restaurants:restaurants.tests.random_text}\pysiglinewithargsret{\code{restaurants.tests.}\bfcode{random\_text}}{\emph{n}}{}
Generates a random text (characters between `a' and `z') with n characters

\end{fulllineitems}

\index{TestClass (class in restaurants.tests)}

\begin{fulllineitems}
\phantomsection\label{restaurants:restaurants.tests.TestClass}\pysiglinewithargsret{\strong{class }\code{restaurants.tests.}\bfcode{TestClass}}{\emph{methodName='runTest'}}{}
Bases: \code{unittest.case.TestCase}

Create an instance of the class that will use the named test
method when executed. Raises a ValueError if the instance does
not have a method with the specified name.
\index{setUp() (restaurants.tests.TestClass method)}

\begin{fulllineitems}
\phantomsection\label{restaurants:restaurants.tests.TestClass.setUp}\pysiglinewithargsret{\bfcode{setUp}}{}{}
Generate random keys (16,24 and 32 bytes)

\end{fulllineitems}

\index{testEncoding() (restaurants.tests.TestClass method)}

\begin{fulllineitems}
\phantomsection\label{restaurants:restaurants.tests.TestClass.testEncoding}\pysiglinewithargsret{\bfcode{testEncoding}}{}{}
Tests the PKCS7Encoder class

\end{fulllineitems}

\index{testSHA() (restaurants.tests.TestClass method)}

\begin{fulllineitems}
\phantomsection\label{restaurants:restaurants.tests.TestClass.testSHA}\pysiglinewithargsret{\bfcode{testSHA}}{}{}
Tests the SHA hashes

\end{fulllineitems}

\index{testHmac() (restaurants.tests.TestClass method)}

\begin{fulllineitems}
\phantomsection\label{restaurants:restaurants.tests.TestClass.testHmac}\pysiglinewithargsret{\bfcode{testHmac}}{}{}
Tests the correction of the `hmac' and `verifies' functions

\end{fulllineitems}

\index{testEncryptCorrection() (restaurants.tests.TestClass method)}

\begin{fulllineitems}
\phantomsection\label{restaurants:restaurants.tests.TestClass.testEncryptCorrection}\pysiglinewithargsret{\bfcode{testEncryptCorrection}}{}{}
Test the correction of the `encrypt' and `decrypt' function, which depends on the cipher and hmac functions
Implicitly, it also tests the parse\_ct\_original function

\end{fulllineitems}


\end{fulllineitems}



\section{restaurants.urls module}
\label{restaurants:restaurants-urls-module}

\section{restaurants.utils module}
\label{restaurants:module-restaurants.utils}\label{restaurants:restaurants-utils-module}\index{restaurants.utils (module)}\index{parse\_ct() (in module restaurants.utils)}

\begin{fulllineitems}
\phantomsection\label{restaurants:restaurants.utils.parse_ct}\pysiglinewithargsret{\code{restaurants.utils.}\bfcode{parse\_ct}}{\emph{ct}}{}
Parses a ciphertext and returns the iv, ciphered message and tag

\end{fulllineitems}

\index{computeSecret() (in module restaurants.utils)}

\begin{fulllineitems}
\phantomsection\label{restaurants:restaurants.utils.computeSecret}\pysiglinewithargsret{\code{restaurants.utils.}\bfcode{computeSecret}}{\emph{g}, \emph{q}}{}
Computes the challenge to be used in the key exchange protocol

\end{fulllineitems}

\index{createChallenge() (in module restaurants.utils)}

\begin{fulllineitems}
\phantomsection\label{restaurants:restaurants.utils.createChallenge}\pysiglinewithargsret{\code{restaurants.utils.}\bfcode{createChallenge}}{}{}
Computes the challenge to be used in the key exchange protocol

\end{fulllineitems}



\section{restaurants.views module}
\label{restaurants:restaurants-views-module}

\section{Module contents}
\label{restaurants:module-restaurants}\label{restaurants:module-contents}\index{restaurants (module)}

\chapter{Indices and tables}
\label{index:indices-and-tables}\begin{itemize}
\item {} 
\emph{genindex}

\item {} 
\emph{modindex}

\item {} 
\emph{search}

\end{itemize}


\renewcommand{\indexname}{Python Module Index}
\begin{theindex}
\def\bigletter#1{{\Large\sffamily#1}\nopagebreak\vspace{1mm}}
\bigletter{r}
\item {\texttt{restaurants}}, \pageref{restaurants:module-restaurants}
\item {\texttt{restaurants.crypto}}, \pageref{restaurants:module-restaurants.crypto}
\item {\texttt{restaurants.key\_exchange}}, \pageref{restaurants:module-restaurants.key_exchange}
\item {\texttt{restaurants.migrations}}, \pageref{restaurants.migrations:module-restaurants.migrations}
\item {\texttt{restaurants.test}}, \pageref{restaurants:module-restaurants.test}
\item {\texttt{restaurants.tests}}, \pageref{restaurants:module-restaurants.tests}
\item {\texttt{restaurants.utils}}, \pageref{restaurants:module-restaurants.utils}
\end{theindex}

\renewcommand{\indexname}{Index}
\printindex
\end{document}
